\chapter{Problemstellung}
\label{chap4}

Nach der kurzen Einführung im ersten Kapitel in die in dieser Arbeit betrachtete Problematik soll im nun folgenden Kapitel die Problemstellung konkretisiert
werden. Dies geschieht auf Grundlage der technischen Gegebenheiten, die in Kapitel \ref{chap2} erläutert wurden, und den bereits existierenden Arbeiten, welche
in Kapitel \ref{chap3} vorgestellt wurden.

In der Einleitung wurde bereits angedeutet, dass zwischen Festplatten und \acp{SSD} ein signifikanter Leistungsunterschied bezüglich der Zugriffszeit besteht.
Die Ursachen hierfür wurden in Kaptitel \ref{chap2} ausführlich dargestellt. Somit wäre es
zwar wünschenswert auf Grund der Leistungsvorteile der \acp{SSD} gegenüber von Festplatten, diese komplett durch \acp{SSD} zu ersetzten. Es würde weiterhin
den Vorteil mit sich bringen, dass sämtliche Software komplett auf \acp{SSD} umgestellt werden könnte. Dies würde sowohl die Leistung der \acp{SSD} ausreizen, da auf
die spezifischen Eigenschaften wie z.B. das \texttt{TRIM}-Kommando eingegangen werden kann, und Entwicklungskosten gespart werden, da nicht für zwei
unterschiedliche Gerätetypen die Software optimiert werden müsste. Aus den preislichen Gegebenheiten, die ebenfalls bereits in Kapitel \ref{chap1} angesprochen und durch die
Tabelle \ref{chap1:preise} verdeutlicht wurden, ist die nahezu vollständige Verdrängung der herkömmlichen Festplatten durch \acp{SSD} in naher Zukunft aber
nicht zu erwarten.

Aus diesem Grund sollte eine Alternative gefunden werden, die besseren Leistungseigenschaften eines \ac{SSD}s effektiv zu nutzen. Hierbei sollten zunächst die
Kostenunterschiede beachtet werden. Daher sollte eine Lösung auch mit vergleichweise wenig \ac{SSD}-Speicher auskommen und trotzdem eine messbare
Leistungssteigerung erbringen. Wie einige der Arbeiten aus Abschnitt \ref{chap3:readyboost} und \ref{chap3:itm} diskutieren, scheint die in der Einleitung in
Abbildung \ref{chap1:mem3} gezeigt Architektur ein Lösungsansatz hierfür zu sein. Dabei wird das \ac{SSD} als Cache genutzt. Dies hat zum einen den Vorteil, dass
der Cache als zusätzliche Schicht in die Speicherhierarchie eingefügt werden kann. Dadurch ist der Anpassungsaufwand an die darüber und darunter liegenden Ebenen
sehr gering. Zum anderen ist der Cache sehr flexibel in Bezug auf den zur Verfügung stehenden Speicher. Aus diesem Grund soll die Nutzung von \acp{SSD} als Cache
für herkömmliche Festplatten in dieser Arbeit detailliert betrachtet werden.

Im Mittelpunkt dieser Arbeit soll der Entwurf, die Implementierung und die Evaluierung eines Linux-Kernelmoduls stehen, welches eine Speicherhierarchie
ermöglicht, wie sie in Abbildung \ref{chap1:mem3} zu sehen ist. Bei der Umsetzung müssen zunächst die momentanen technischen Rahmenbedingungen beachtet werden.
Zur Zeit gebräuchliche Festplatten haben eine Kapazität von bis zu zwei Terabyte. Aus diesem Grund sollte der Cache mindestens diese Größe unterstützen, wobei
eine Größe um zehn Terabyte wünschenswert ist, um in den nächsten zwei Jahren nicht an die Grenze der Möglichkeiten des Caches zu stoßen. Ähnliches gilt für die
Größe des verwendeten \ac{SSD}. Ausgehend von der Vermutung, dass das Cachegerät wesentlich kleiner ist als die Festplatte, da es ansonsten als alleiniges
Speichergerät verwendet werden könnte, sollte der Cache eine Größe von bis zu einem Terabyte beim Cachegerät unterstützen.

Des Weiteren muss die Entscheidung getroffen werden, ob der Cache block- oder dateisystemorientiert arbeiten soll. Diese Frage ist grundlegender Natur, da auf
der Dateisystemebene wesentlich mehr Informationen über Daten zur Verfügung stehen als auf der Blockebene. Dieses beeinflusst nicht nur den Aufbau des
Kernelmoduls an sich sondern auch die Möglichkeiten bezüglich der Cachingstrategien. Ein weiterer Punkt, den es zu beachten gilt, ist die Portabilität. Ein
blockorientierter Cache könnte mit einem beliebigen Dateisystem genutzt werden. Ein dateisystemorientierter Cache wäre an das zugrunde liegende Dateisystem
gebunden und müsste für jedes neue Dateisystem spezifisch implementiert werden. Unter anderem auf Grund dieser Tatsache soll der Cache blockorientiert entwickelt
werden. Diese Entscheidung wird dadurch unterstützt, dass mit dem in Abschnitt \ref{chap3:zfs} vorgestellten Arbeiten von \textcite{zfs1,zfs2} untersuchten
ZFS bereits eine effiziente Implementierung eine Caches auf Dateisystemebene existiert.

Der Cache soll weiterhin die Randbedingung erfüllen, möglichst ressourcenschonend zu arbeiten. Die Nutzung eines Caches ist nur sinnvoll, wenn durch die Nutzung
andere Teile des Systems nicht massiv  verlangsamt werden. Ein besonderes Augenmerk liegt hierbei auf dem Speicherverbrauch und der Prozessorleistung. Der Cache
wird auf Grund von Erwägungen bezüglich der Geschwindigkeit große Teile seiner Verwaltungsdaten im Arbeitsspeicher halten müssen. Dadurch steht weniger Speicher
für die Nutzung durch Programme bzw. das Cachen von Daten zur Verfügung. Sollte der Cache sehr viel Speicher für die Metadaten benötigen, könnte es zu massiven
Leistungseinbußen kommen. Aus diesem Grund soll der zu implementierende Cache möglichst geringen Speicherverbrauch in Abhängigkeit von der Cachegröße aufweisen.
Ähnliches gilt für die Nutzung des Prozessors. Auch hier sollte darauf geachtet werden, dass die Prozessorlast durch den Cache möglichst gering ist. Sollte dies
nicht der Fall sein, würde es in einem Multitasking-System zu einer Verlangsamung anderer Prozesse kommen. Dies ist nicht erwünscht.

Neben dem ressourcenschonenden Arbeiten sollte der Cache weiterhin einfach zu managen sein. Konkret heißt dies, dass Cachegeräte jederzeit in ein System
eingefügt und entfernt werden können ohne größeres Hinzutun des Nutzers. Somit sollte der Cache zunächst mit einigen wenigen, wenn nicht sogar nur einem
Kommando aufgebaut werden können. Weiterhin sollte das Entfernen des Cachegeräts nicht zu einem Ausfall des Systems führen und auch kein Eingreifen des Benutzers
erfordern.

Schließlich sollte es auch möglich sein, von dem gecachten Gerät zu booten. Auch hierbei sollte der Cache möglichst einfach in ein bestehendes System eingefügt
werden können.

Es lassen sich somit folgende Anforderungen an den Cache feststellen:

\begin{itemize}
  \item Blockorientiert
  \item Ressourcenschonend
  \item Einfaches Management
  \item Bootfähig
\end{itemize}

Neben der Umsetzung der technischen Eigenschaften soll ein geeigneter Cache-Algorithmus gefunden werden, der zum einen gute Ergebnisse bezüglich der Leistung
verspricht, aber auch durch entsprechende Parameter bezüglich der obigen Kriterien optimiert werden kann. Hierbei sollen unter anderem die Ergebnisse von 
\textcite{cache1} beachtet werden, welcher in Abschnitt \ref{chap3:ersetzen} vorgestellt wurden.

Nach der erfolgten Beispielimplementierung soll diese auf die Erfüllung der soeben genannten Anforderungen geprüft werden. Dies soll anhand zweier Szenarien
stattfinden. Zum einen soll ein reales System betrachtet werden. Dabei sollen die Unterschiede bei Leistung und Gerätenutzung in Abhängigkeit vom genutzten
Systemspeicher genauer betrachtet werden. Hierbei soll eine durch ein \ac{SSD} gecachte Festplatte untersucht werden und mit einer einezlnen Festplatte bzw.
einem \ac{SSD} verglichen werden. Zum anderen soll ein synthetisches Testszenario, welches ein Desktopsystem simuliert, entworfen werden. Dies soll es
ermöglichen, in einem vertretbaren Zeitrahmen mehrere Arbeitstage zu simulieren. Des Weiteren soll es den Vergleich verschiedener Massenspeichergeräte unabhängig
vom genutzten Betriebs- bzw. Dateisystem ermöglichen. Dies soll ein Vergleich des blockbasierten Caches für Linux mit ZFS ermöglicht, welches in der
Referenzimplementierung nicht für Linux zur Verfügung steht. Es soll die Referenzimplementierung, welche z.B. unter OpenSolaris \cite{solaris} zur Verfügung
steht, genutzt werden, um schlechte Leistungswerte auf Grund einer mangelhaften Implementierung vorzubeugen.
