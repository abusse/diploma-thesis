\chapter{Verwandte Arbeiten}
\label{chap3}

Vor dem Formulieren einer konkreten Aufgabenstellung gilt es, sich mit Arbeiten auseinander zu setzen, die sich zum einen mit Caching und zum anderen mit dem
Zusammenspiel von Flash- und Magnetspeicher beschäftigen. Dazu werden in diesem Kapitel verschiedene Konzepte betrachtet. Abschnitt \ref{chap3:zfs}
thematisiert das von Sun Microsystems entwickelte ZFS, bei dem eine Cacheimplementierung auf der Dateisystemebene vorliegt. Unter Punkt \ref{chap3:readyboost}
wird die proprietäre ReadyBoost-Technologie von Microsoft betrachtet, welche die Nutzung eines USB-Flash-Speichermediums als Cache vorsieht. Im Anschluss daran
wird in Abschnitt \ref{chap3:itm} auf eine flashbasierte Cacheimplementierung der Firma Intel eingegangen, welche ein dediziertes Hardwaremodul nutzt. In
Abschnitt \ref{chap3:griffin} wird ein Artikel angesprochen, in dem der gegenteilige Ansatz gewählt wird und ein \ac{SSD} durch eine Festplatte gecacht wird. Die
Erwähnung dieses Artikels soll verdeutlichen, dass auch das nahezu komplett gegenteilige Vorgehen erforscht wird. In Abschitt \ref{chap3:dm-cache} wird auf eine
Forschungsarbeit eingegangen, die das Cachen von Netzwerkdaten auf Festplatten betrachtet. Dies ist für diese Arbeit von Relevanz, da im Artikel zum einen von
Blockgeräten ausgegangen wird und zum anderen ein ähnliches Problem angesprochen wird, was in einer tieferen Ebenen der Speicherhierarchie liegt. Zum
Abschluss dieses Kapitels wird in Abschnitt \ref{chap3:ersetzen} ein Artikel betrachtet, welcher sich grundsätzlich mit Ersetzungsstrategien für Caches
beschäftigt. Die Auswertung dieses Artikels ist wichtig, da die Wahl einer geeigneten Strategie von erheblicher Bedeutung für diese Arbeit ist.

\section{Sun Microsystems' ZFS}
\label{chap3:zfs}

ZFS ist ein von Sun Microsystems entwickeltes Dateisystem, welches primär für die Verwendung in Server- und Rechenzentren ausgelegt ist. Es ist kein
klassisches Dateisystem, welches auf die Verwaltung von Daten auf einem einzelnen logischen oder physischen Gerät beschränkt ist, sondern eines, das die
Verwaltung von Verbänden von Blockgeräten ermöglicht. ZFS organisiert die Blockgeräte hierfür in so genannten Pools, wobei diese als ausfallsicherheitssteigernde
\acp{RAID} organisiert werden können. Sollten ausreichend Geräte zur Verfügung stehen, werden zur Steigerung der Geschwindigkeit die Daten auf den Geräten
parallel verteilt. Geräte können hierbei jederzeit zu einem Pool hinzugefügt werden, um den Speicherplatz, die Datensicherheit oder die Leistung zu erhöhen.

Die Artikel von \textcite{zfs1} und \textcite{zfs2} setzen sich mit einem weiteren Mechanismus von ZFS auseinander. Sie betrachten die Möglichkeit, ein
Blockgerät als Cache zu benutzen. Ein Blockgerät kann hierfür zu einem Pool hinzugefügt werden und als Cache markiert werden, wobei diese Blockgeräte
performanter sein sollten als die zu cachenden Geräte. Aus diesem Grund kommen hierfür z.B. \acp{SSD} in Frage.

\Citeauthor{zfs2} vergleicht nun die Zugriffszeiten auf das ZFS-Dateisystem in Abhängigkeit davon, ob die Anfrage aus dem Arbeitsspeicher des Systems,
des \ac{SSD}s oder der Festplatte bedient wird. Die Ergebnisse finden sich im Diagramm in Abbildung \ref{zfs2}. Es ist klar zu erkennen, dass die Zugriffszeiten
um Größenordnungen auseinander liegen. Somit wird deutlich, dass die Nutzung von \acp{SSD} als zusätzlicher Cache aus Sicht der Leistung sinnvoll ist.

\begin{figure}[h]\centering
	\includetikz{figures/chapter3/zfs}%
	\caption[Zugriffszeit auf den ZFS Pool in Abhängigkeit vom bedienenden Systemteil]{Zugriffszeit auf den ZFS Pool in Abhängigkeit vom bedienenden Systemteil
	(nach \textcite{zfs2})}
	\label{zfs2}
\end{figure}

Der Artikel von \textcite{zfs1} setzt sich mit dem ökonomischen Aspekt der Nutzung von \acp{SSD} als Cache auseinander. Es wird durch eine Beispielrechnung
veranschaulicht, wie groß die Ersparnis in Anschaffung und Stromverbrauch und somit indirekt der Betriebskosten durch die Nutzung von \acp{SSD} als Cache für ein
gegebenes Szenario ist. Der Artikel kommt dabei zu dem Schluss, dass die Anschaffung des \ac{SSD} gestützten Systems bei gleichbleibender Leistung 64\%
günstiger und der Stromverbrauch 77\% geringer ist. Dabei gilt es jedoch anzumerken, dass für den Vergleich Serverfestplatten genutzt wurden, welche einen
Verbrauch von 17,5 Watt hatten. Bei der Nutzung von Desktop- oder Notebookfestplatten mit einem Verbrauch von ca. 3-8 Watt wäre der Unterschied geringer
ausgefallen, jedoch hätte es wahrscheinlich einen deutlichen Leistungsunterschied gegeben.

\section{Microsofts ReadyBoost}
\label{chap3:readyboost}

Im Patent \cite{patent:20060090031} wird eine Technik zum Cachen von Plattensektoren auf einem Wechselmedium beschrieben. Diese Technik wird von Microsoft unter
der Bezeichnung \textit{ReadyBoost} vermarktet. Die Technik dient in erster Linie zur Leistungssteigerung älterer Systeme, die eine geringe
Menge an Hauptspeicher aufweisen. Bei diesen Systemen ist das Cachen von Daten im Hauptspeicher, wie in der Einleitung beschrieben wurde, kaum möglich, so dass
ein intensives Arbeiten auf dem Plattengerät erforderlich ist. Das Patent beschreibt hierbei die Möglichkeit, diesen Umstand zu mildern, indem z.B. ein günstiges
USB-Speichermedium zum Zwischenspeichern der Daten eingesetzt wird. Dies spiegelt nochmals die veränderte Speicherhierarchie wieder, welche in der Einleitung
gezeigt wurde. Die Ausarbeitung kommt dabei zu teilweise sehr hohen Leistungssteigerungen, wie beispielhaft der Tabelle \ref{table:readyboost} entnommen
werden kann.

Hierbei wurden die Zugriffsmuster typischer Desktopsysteme simuliert. Dabei wurden als Cache USB-Speichermedien mit verschiedenen Größen eingesetzt. Es ist zu
erkennen, dass die Lesezeit des Datensatzes sich auf dem mit 512MB Arbeitsspeicher ausgestatteten System 1 auf 18\% reduzierte.

\begin{table}[t]\centering\small
	\newcolumntype{C}[1]{>{\centering\arraybackslash}m{#1}}
	\caption{Geschwindigkeitsgewinn für Endnutzersysteme bei der Nutzung von ReadyBoost (Quelle: \textcite{patent:20060090031})}\vspace{0.25cm}
  \begin{tabular}{lC{2.5cm}cccccc}
	System & Lesezeit in Sek. der simulierten HDD & \multicolumn{6}{C{8cm}}{Prozentsatz der simulierten Lesezeit mit einem USB2-Speichermedium der Größe:}\\
	 &  & 0MB & 32MB & 64MB & 128MB & 256MB & 512MB \\ \hline
	System 1 (128MB) & 1259 & 100\% & 89\% & 70\% & 37\% & 18\% & 18\% \\
	System 2 (128MB) & 1011 & 100\% & 90\% & 70\% & 38\% & 22\% & 22\% \\
	System 3 (128MB) & 2159 & 100\% & 88\% & 72\% & 44\% & 25\% & 20\% \\
	System 4 (128MB) &  866 & 100\% & 90\% & 80\% & 63\% & 48\% & 37\% \\
	System 5 (256MB) & 1747 & 100\% & 92\% & 85\% & 70\% & 52\% & 40\% \\
	System 6 (256MB) & 2187 & 100\% & 94\% & 87\% & 76\% & 66\% & 57\% \\
	\end{tabular}
	\label{table:readyboost}
\end{table}

\section{Intels Turbo Memory}
\label{chap3:itm}

\Textcite{intel:turbo} beschreiben die \ac{ITM} Technologie. Bei \ac{ITM} handelt es sich um ein Hard-/Software-Codesign. Auf der
Hardwareseite finden sich ein NAND-Flash-Speicher, ein Flash-Controller und ein sogenanntes OROM, welches im Fehlerfall aktiv wird. Das Hardwaremodul wird über
den PCI-Express-Bus an das System angebunden. Auf der Softwareseite befindet sich der \ac{ITM}-Treiber, welcher die Ansteuerung des Hardwaremoduls übernimmt, die
Schnittstelle für das Betriebssystem zur Verfügung stellt und Funktionen, die die Nutzung des Hardwaremoduls zum einen als \ac{SSD} zulassen und zum anderen als
Cache für Festplatten. Dieser Sachverhalt ist in Abbildung \ref{chap3:intel1} zu sehen.

Bei \ac{ITM} handelt es sich um einen Lese-/Schreibcache, der auch nach einem Neustart erhalten bleibt. Aus diesem Grund sind besondere Maßnahmen für die
Ausfallsicherheit zu treffen, da die aktuellen Daten nicht einheitlich auf einem Gerät zu finden sind, sondern auf der Festplatte und dem Cache verteilt sind.
Intel hat den bereits von \textcite{dm-cache} angesprochenen Ansatz gewählt, die Metadaten des Caches sowohl im Arbeitsspeicher als auch im Flashspeicher vorzuhalten.
Für die Verwaltung der Metadaten werden pro vier Kilobyte Cacheblock 40 Byte Daten im Hauptspeicher benötigt. Es liegt also ein Verhältnis von Speicherverbrauch
zu Cachegröße von ca. 1:100 vor. Weiterhin wird auf das Problem hingewiesen, dass durch die Einführung eines persistenten Caches gleichzeitig zwischen zwei
persistenten Speicherebenen eine nicht-persistente in Form des Festplattencaches eingefügt wird (Abbildung \ref{chap3:intel2}). Dadurch erhöht sich die Gefahr von
Datenverlust bei einem Systemfehler.

\begin{figure}[t!]\centering
  \includetikz[0.8]{figures/chapter3/itm1}%
  \caption[Blockdiagramm des Intel Turbo Memory]{Blockdiagramm des Intel Turbo Memory (nach \textcite{intel:turbo})}
  \label{chap3:intel1}
\end{figure}

\begin{figure}[b!]\centering
	\includetikz[0.85]{figures/chapter3/itm2}%
	\caption[Speicherhierarchie mit Intel Turbo Memory]{Speicherhierarchie mit Intel Turbo Memory (nach \textcite{intel:turbo})}
	\label{chap3:intel2}
\end{figure}

Dies kann zu Problemen im Fehlerfall führen, wenn das Schreiben der Daten vom Cache an das Betriebssystem bereits bestätigt wurde, aber noch nicht sicher ist,
ob der Schreibvorgang zwischen Cache und Festplatte korrekt abgeschlossen wurde. Zur Lösung dieses Problems wird im Artikel vorgeschlagen, den Festplattencontroller
vor der Aktualisierung der Metadaten des Caches explizit anzuweisen, den Festplattencache zu leeren.

Des Weiteren wird auf das Problem der Separierung von Festplatte und Cache hingewiesen, was vor allem bei einem Write-Back-Cache auftritt. Um das
Problem zu lösen, werden beim Herunterfahren des Rechners grundsätzlich alle als verändert markierten Blöcke des Caches auf die Festplatte zurückgeschrieben.
Sollte dies durch einen Systemfehler nicht möglich sein, werden die Daten beim Starten des Rechners durch das OROM zurück auf die Festplatte geschrieben, ehe der
normale Bootvorgang startet. Diese Funktionalität des \ac{ITM} ist der Grund dafür, dass es sich bei \ac{ITM} um ein Hard-/Software-Codesign handelt. Die Daten
werden unabhängig vom Gerätetreiber selbständig von der Hardware kopiert. Sollten Festplatte und Cache jedoch nach einem Systemfehler separiert werden, sind
besondere Maßnahmen zu treffen. Von den Autoren wird hierfür auf Patent~\cite{patent:20070233947} verwiesen.

Weiterhin wird in dem Artikel auf Caching-Strategien eingegangen. Es wird ausgeführt, dass es wenig sinnvoll ist, lange Lese-/Schreib-Sequenzen zu cachen. Als
effektiver wird angesehen, die Daten zu cachen, die ungeordnet angefragt werden. Um festzustellen, ob Daten zu der einen oder anderen Art gehören, benutzen die
Autoren die Größe der jeweiligen Datenanfrage. Außerdem sind sie der Meinung, es sei nur sinnvoll Daten zu cachen, die mehr als einmal wiederbenutzt werden. Von
den Autoren wird hierbei darauf hingewiesen, dass dies besonders beim Pagefile\footnote{Bildet die Grundlage des virtuellen Speichers eines
Betriebssystems und enthält die ausgelagerten Seiten.} des Betriebssystems der Fall ist.

Die Autoren weisen in weiteren Teilen des Artikels außerdem darauf hin, dass die Hit-Rate\footnote{Ist das Verhältnis zwischen der Gesamtzahl der Anfragen
und den aus dem Cache bedienten Anfragen.} kein gutes Kriterium ist, um die Qualität und Leistung des Caches zu beurteilen. Sinnvoller sei es, die Länge der
aufeinanderfolgenden Anfragen zu messen, welche nicht von der Festplatte sondern vom Cache bedient werden können.

Die Autoren dieses Artikels kommen zu dem Schluss, dass das Nutzen von Flash-Speicher zum Cachen von Festplatten sinnvoll ist. Ein Auszug ihrer Ergebnisse
bei der Nutzung von \ac{ITM} findet sich in Tabelle \ref{chap3:intel:table}.

\begin{table}[H]\centering\small
	\newcolumntype{C}[1]{>{\centering\arraybackslash}m{#1}}
	\caption{Ergebnisse der Nutzung von ITM (Quelle: \textcite{intel:turbo})} \vspace{0.25cm}
  \begin{tabular}{p{9cm}|l}
	\multicolumn{1}{c|}{\textbf{Art der Systemnutzung}} & \multicolumn{1}{c}{\textbf{Leistungsgewinn}} \\ \hline
	Arbeit mit Microsoft Office mit einer speicherintensiven Anwendung im Hintergrund & Verdoppelte Reaktionszeit \\ \hline
	Bootzeit bis zum Startmenü & 22\% schneller \\ \hline
	PCMark05 disk & 20\%ige Verbesserung \\ \hline
	Ein Vorbeiflug an einem Nationalpark mit Google Earth gefolgt von der Nutzung von Adobe Photoshop Elements 5.0 zur Erstellung einer Slideshow, die Bilder
	desselben Parks zeigt & 2-fache Geschwindigkeit \\  \hline
	\end{tabular}
	\label{chap3:intel:table}
\end{table}

\section{Griffin}
\label{chap3:griffin}

Das von \textcite{griffin} vorgeschlagene Verfahren nutzt eine Festplatte zum Cachen von \ac{SSD}-Zugriffen, um die Anzahl der Schreibvorgänge auf das \ac{SSD} zu
verringern. Der Artikel hat auf den ersten Blick wenig mit dem in dieser Arbeit behandelten Thema gemein, ist aber für einige Designentscheidungen trotz alledem
von Bedeutung. Von den Autoren wird auf das Problem eingegangen, dass bei \acp{SSD} die Anzahl der möglichen Schreibvorgänge begrenzt ist. Aus diesem Grund wird
vorgeschlagen, statt Schreibvorgänge direkt auf einem \ac{SSD} auszuführen, diese auf einer Festplatte zu cachen. Leistungseinbußen werden hierbei durch die
Architektur des Caches vermieden. Der Cache wird hierfür in Form eines Logs aufgebaut, so dass neue Schreibvorgänge stets an das Ende des Caches geschrieben
werden und somit der ggf. ungeordnete Schreibzugriff auf dem \ac{SSD} in einen geordneten auf der Festplatte überführt wird. Da ungeordnete Schreibzugriffe auf
\acp{SSD} meist langsamer sind als geordnete auf einer Festplatte, sollte es laut der Autoren kaum Leistungsunterschiede geben. Des Weiteren wird der
Kostenvorteil einer Festplatte gegenüber einem \ac{SSD} genutzt. Durch den Kostenvorteil ist es möglich, dass der Cache größer ist als das Quellgerät.
Abschließend gehen die Autoren näher auf die von ihnen ausgeführten Tests ein, die neben einer Festplatte als Cache auch das Cachen von \ac{MLC}-\acp{SSD} mit
Hilfe von \ac{SLC}-\acp{SSD} betrachten.

\section{Dynamic Policy Disk Caching for Storage Networking}
\label{chap3:dm-cache}

Die Arbeit von \textcite{dm-cache} beschäftigt sich mit dem Cachen von \acp{SAN} auf lokalen Festplatten. Das Thema des Reports hat auf den ersten Blick nichts mit dem
Thema dieser Arbeit gemein, aber es wird auf die Eigenschaft eines \ac{SAN}s eingegangen, blockbasierend zu sein. Somit wird im Report das Cachen eines
Blockgeräts durch ein anderes beschrieben, was beim Cachen eines \ac{SSD} durch eine Festplatte ebenfalls der Fall ist. Weiterhin ist der Unterschied bezüglich der Zugriffszeit
zwischen Festplatte und \acp{SAN} vergleichbar mit der zwischen \acp{SSD} und Festplatten. Im Report wird eine Linux-Kernelerweiterung beschrieben, die die
Aufgabe des Cachens erfüllt. Da die Erweiterung generell auf logische Blockgeräte zugeschnitten ist, ist sie auch für diese Arbeit relevant und soll somit untersucht
werden.

Schließlich kommt der Report in Bezug auf \acp{SAN} zu dem Schluss, dass unter der Verwendung der implementierten Erweiterung ein beachtlicher Leistungsgewinn
zu beobachten ist. Die Laufzeiten der gewählten Benchmarks verringern sich auf bis zu einem Drittel. Weiterhin wird die Ausfallsicherheit des Caches diskutiert.
Diese Diskussion ist auch für diese Arbeit von Bedeutung. Für die Ausfallsicherheit ist hauptsächlich das Speichern der Metadaten maßgebend. Die gecachten
Inhalte liegen ohnehin in einem persistenten Speicher, so dass sie beim Ausfall nicht verloren gehen können. Anders sieht es bei den Metadaten aus, da diese aus
Leistungsgründen im nicht persistenten Arbeitsspeicher gehalten werden und bei einem Ausfall verloren gehen.

\Citeauthor{dm-cache} stellen drei Lösungsmöglichkeiten vor, wobei nur die letzen beiden auch für einen Write-Back-Cache geeignet sind. Die erste Möglichkeit
besteht im kompletten Neuaufbau des Caches. Die Zweite im zusätzlichen Speichern der Metadaten im persistenten Speicher. Die dritte Möglichkeit ist eine
Kombination aus den ersten beiden, wobei nur Daten von Cacheblöcken gespeichert werden, die lediglich im Cache aktuell sind.

\section{Ersetzungsstrategien für Caches}
\label{chap3:ersetzen}

Da ein Cache für gewöhnlich weniger Speicherplatz aufweist als das Quellmedium\footnote{Dies ist das gecachte Medium.}, ist die Frage, welche Daten im Cache
gehalten werden und welche nicht, von zentraler Bedeutung. Genauer gesagt ist es die Frage, welcher Block bei einem vollen Cache als nächster ersetzt wird. Die
optimale Vorgehensweise hierfür ist im Verfahren von \textcite{cache1} beschrieben. Es besagt, dass der Block verdrängt wird, auf den in der Zukunft am
längsten nicht mehr zugegriffen wird. Da dieses Verfahren jedoch Wissen über das zukünftige Verhalten des Systems voraussetzt, es ist nur bei einer
Offline-Planung\footnote{Das gesamte Verhalten des Systems ist während der Entwurfszeit bekannt.}  einsetzbar. Für eine Online-Planung muss versucht werden, das
zukünftige Verhalten des Systems vorherzusagen. Die Vorhersage geschieht für gewöhnlich auf Grundlage des vergangenen Verhaltens des Systems. Hierbei liegt die
Annahme zu Grunde, dass das zukünftige Verhalten dem vergangenen Verhalten ähnelt. Laut \textcite{cache2} gibt es hierbei zwei grundsätzliche Herangehensweisen --
das \textit{\ac{LRU}}-Prinzip und das \textit{\ac{LFU}}-Prinzip.

Das \ac{LRU}-Prinzip basiert auf der Annahme, dass auf einen Block, auf den bereits sehr lange nicht mehr zugegriffen wurde, auch in Zukunft in absehbarer Zeit
nicht zugegriffen wird. Dieses Verfahren bedeutet einen relativ geringen Aufwand, da das System lediglich den am längsten nicht genutzten Block bestimmen muss.
Dies kann z.B. durch eine Liste geschehen, in der bei einem Zugriff auf einen Block dieser immer an den Anfang gesetzt wird. So würde sich der am längsten nicht
genutzte Block am Ende befinden.

Das \ac{LFU}-Prinzip hingegen basiert auf der Zugriffshäufigkeit bzw. Zugriffsfrequenz. Es geht von der Annahme aus, dass Blöcke, die in der Vergangenheit häufig
benutzt wurden, auch in der Zukunft häufig benutzt werden. Dieses Verfahren hat im Vergleich zum \ac{LRU}-Algorithmus einen höheren Aufwand, da zu jedem
Cacheblock Zugriffsstatistiken geführt werden müssen und die Blöcke nach ihrer Zugriffsfrequenz sortiert werden müssen. Zudem besteht das Problem, dass
Cacheblöcke, auf die in der Vergangenheit oft zugegriffen wurde, lange im Cache verbleiben, auch wenn sie nicht mehr benötigt werden. Ein Beispiel hierfür könnte
eine gelöschte Datei sein, die zwar in der Vergangenheit oft genutzt wurde, jetzt jedoch nicht mehr existiert und somit die Daten auch im Cache nicht mehr
benötigt werden.

Von \citeauthor{cache2} wird nun ein Raum von Caching-Strategien postuliert, der von den beiden genannten Strategien aufgespannt wird und diese als Extremfälle
beinhaltet. Es werden dabei Strategien untersucht, die in diesem Raum liegen und mit den Extremfällen und der optimalen ex-post Lösung verglichen werden. Die
Autoren kommen zu dem Schluss, dass Strategien, die zwischen \ac{LRU} und \ac{LFU} liegen, meist besser geeignet sind. Weiterhin lassen sich diese flexibel
implementieren, so dass sich die Strategien durch einen einzigen Parameter Richtung \ac{LRU} oder \ac{LFU} steuern lassen und für ein gegebenes Szenario leicht
optimieren lassen.
