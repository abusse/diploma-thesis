%% Temporarily enlarge this page to push
%% down the bottom margin
\pagestyle{empty}
\mbox{}\clearpage{}
\vspace*{5mm}
\begin{center}
\begin{minipage}{.8\textwidth}
\titleFont \large
\Author \newline
\titleFontBold \Large
\Title \newline
\titleFont \normalsize

\setlength{\parskip}{1em}

Diese Arbeit beschäftigt sich mit der Frage, ob es möglich und sinnvoll ist ein
Solid State Drive (SSD) als Cache für eine herkömmliche Magnetscheiben-basierte
Festplatte zu nutzen.

In aktuellen Rechnersystemen stellt häufig der Massenspeicher in Form einer
Festplatte den Flaschenhals des Systems dar. Dies ist darauf zurückzuführen,
dass Festplatten auf Grund ihres mechanischen Aufbaus eine Zugriffszeit haben,
die um mehrere Größenordnungen schlechter ist als die der nächsten Stufe in der
Speicherhierarchie -- dem Arbeitsspeicher. Um dieses Problem zu beseitigen,
wurde in den letzten Jahren halbleiterbasierter Massenspeicher als Ersatz für
Festplatten eingeführt, der dieses Defizit nicht besitzt. Diese Laufwerke werden
unter der Bezeichnung SSD vermarktet. SSDs haben jedoch den Nachteil, dass die
Kosten pro Byte wesentlich über denen von Festplatten liegen. Darum ist mit
einer vollständigen Substitution von Festplatten durch SSDs in den kommenden
Jahren kaum zu rechnen.

Die momentane Situation, die daraus resultiert, ist die, dass Anwender häufig
genutzte Daten auf einer meist kleinen SSD speichern und die restlichen Daten
auf einer langsameren Festplatte. Dieses Vorgehen ist für den Nutzer jedoch sehr
umständlich. Deshalb wird in dieser Arbeit die Nutzung von SSDs als
transparenter Cache für Festplatten vorgeschlagen. Dadurch würde nur ein
geringes Eingreifen des Nutzers erforderlich sein und es ihm trotzdem
ermöglicht, die Vorteile von SSDs zu nutzen.

Es werden im Verlauf dieser Arbeit dafür zunächst die technischen Grundlagen von
Festplatten und SSDs dargestellt und andere Arbeiten betrachtet, die ein
ähnliches Konzept verfolgen bzw. für die praktische Realisierung des Cache von
Bedeutung sind. Auf Grundlage dieser technischen und theoretischen
Rahmenbedingungen wird eine konkrete Problem- und Aufgabenstellung formuliert.
Basierend auf dieser wird eine Architektur für einen blockbasierten Cache
vorgeschlagen, deren konkrete Implementierung ebenfalls in dieser Arbeit
beispielhaft dargestellt wird. Mit Hilfe dieser Beispielimplementierung wurden
für diese Arbeit Simulationen und Messungen durchgeführt. Sie ermöglichen es die
Frage zu beantworten, ob es sinnvoll ist, eine SSD als Cache zu nutzen. Somit
wird abschließend diese Fragestellung anhand der gewonnenen Messergebnisse unter
den Gesichtspunkten der Leistungssteigerung und des zusätzlichen
Ressourcenverbrauchs durch den Cache diskutiert.

\end{minipage}
\end{center}
