\chapter{Zusammenfassung und Ausblick}
\label{chap8}

Zusammenfassend kann konstatiert werden, dass die Erweiterung der Speicherhierarchie, wie sie in der Einleitung dargestellt wurde, realisierbar und sinnvoll ist.
Es wurde gezeigt, dass eine Implementierung auf Blockebene mindestens dieselbe Leistung erbringt wie eine Implementierung auf Dateisystemebene. Auf Grund der
universelleren Einsetzbarkeit sollte die blockbasierte Implementierung vorgezogen werden und die Entwicklung weiter vorangetrieben werden.

Auf systemspezifischer Ebene des Linux Betriebssystems hat sich die Nutzung des \ac{DM}-Framework als solide und effizient herausgestellt. Des Weiteren wurde
gezeigt, dass es prinzipiell möglich ist, von einem in dieser Arbeit entworfenen Cache zu booten. Da diese Funktionalität jedoch eine Kernelerweiterung nutzt,
von der momentan nicht klar ist, ob sie in den offiziellen Kernel einfließt, ist zur Zeit auch unklar, ob der beschrittene Weg der richtige ist. Sollte die
Erweiterung in den Kernel aufgenommen werden, ist diese Frage ganz klar mit ja zu beantworten, sollte sie jedoch nicht aufgenommen werden, sollte auch der Weg
über eine moduleigene Lösung betrachtet werden. Ähnliches gilt für das in Abschnitt \ref{chap6:boot} angesprochene Problem beim Neustart. Auch hier sollte
zunächst versucht werden, eine einheitliche Lösung durch Erweiterung des \ac{DM}-Frameworks zu finden. Sollte dies scheitern, muss auch hier der Ansatz der
modulinternen Lösung weiterverfolgt werden.

Bei der weiteren Erforschung des Themenbereichs ist es notwendig eine Reihe von Problemen auf der Seite der Implementierung zu lösen. Zunächst gilt es
die Synchronisationsprobleme der Write-Back- und Write-Hybrid-Strategie zu betrachten. Hierfür bedarf es einer detaillierten Analyse des bestehenden Codes in
Hinblick auf Konkurrenzsituationen und mögliche Deadlock-Gefahren. Nachdem dieses Problem gelöst ist, müssen die Leistungsdefizite des Write-Back-Caches
beseitigt werden. Dies könnte in der Art geschehen, dass in die \texttt{map}-Funktion ein Filter eingefügt wird, der sequenzielle Schreibzugriffe erkennt und
entsprechende Anfragen direkt an das Quellgerät weiterleitet. Die Implementierung des Filters wird dabei vor die Herausforderung gestellt, dass die
\texttt{map}-Funktion nur Anfragen in Größe der Cacheblockgröße erhält. Darum ist es problematisch die Gesamtgröße einer Anfrage zu bestimmen.

Nach der erfolgreichen Implementierung des Write-Back-Caches muss unbedingt die Sicherung der Cachemetadaten verbessert werden. Bei einem Write-Through-Cache
kann beim Verlust der Metadaten der Cache problemlos neu aufgebaut werden. Beim Write-Back- und Write-Hybrid-Cache ist dies nicht möglich, da eine Menge
gültiger Daten exklusiv auf dem Cachegerät vorliegen. Demzufolge ist es schwierig die Metainformationen über diese Daten nicht zu verlieren. Die momentane
Implementierung nimmt auf diesen Sachverhalt keinerlei Rücksicht, so dass hierbei eine weitere Analyse notwendig ist.

Nach der Behebung der Implementierungsdefizite sollte die Leistungsmessung verbessert werden. Es wäre wünschenswert ähnlich wie in den Arbeiten aus Kapitel
\ref{chap3} die Leistungsbewertung anhand realer Zugriffe durchzuführen. Hierfür müsten zunächst Blockgerätzugriffe auf Produktivsystemen aufgezeichnet werden.
Anschließend sollten diese auf den implementierten Cache angewendet werden. Da gerade Letzteres sehr zeitaufwendig ist, kann die in dieser Arbeit verwendete
Messungsmethode als Vorauswahl für vielversprechende Konfigurationen genutzt werden. Diese können dann in einer länger andauernden Testreihe mit den
aufgezeichneten Realdaten evaluiert werden.
