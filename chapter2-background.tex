\chapter{Technisches Umfeld}
\label{chap2}

Um einen in der obigen Art beschriebenen Cache realisieren zu können, muss sich zunächst mit der genauen Funktionsweise der beteiligten Geräte
auseinandergesetzt werden. Dies sind im konkreten Fall herkömmliche magnetscheibenbasierte Festplatten und Flashspeicher basierte \acp{SSD}. Diese haben jeweils
spezifische Vor- aber auch Nachteile. Aus diesem Grund wird in diesem Kapitel detailliert auf den Aufbau und die Funktionsweise beider Gerätetypen eingegangen.
Festplatten werden hierbei in Abschnitt \ref{chap2:hdd} und \acp{SSD} in Abschnitt \ref{chap2:ssd} betrachtet.

\section{Festplattenlaufwerk (HDD)}
\label{chap2:hdd}

Der Aufbau und die Funktionsweise herkömmlicher Festplatten wird unter anderem von \textcite{info2} und \textcite{tanenbaum1} genauer beschrieben. Eine Festplatte
arbeitet nach dem Prinzip, Information in Form der Ausrichtung von magnetischen Teilchen zu speichern.

\begin{figure}[b!]\centering
    \includetikz[0.4]{figures/chapter2/hdd1}%
    \caption[Schematische Darstellung einer Festplatte]{Schematische Darstellung einer Festplatte (nach \textcite[101]{tanenbaum1})}
    \label{img:hdd1}
\end{figure}

Sie bestehen physisch aus einer oder mehreren ferromagnetisch beschichteten Scheiben, auch Platter genannt. Diese können sowohl ein- als auch beidseitig
beschichtet sein. Sie sind an einer Achse angebracht, welche mit Hilfe eines Motors angetrieben wird. Weiterhin gibt es einen Lesearm, der in eine beliebige
radiale Position über den Plattern gebracht werden kann. An der Spitze des Armes befindet sich der Lese-/Schreibkopf. Außerdem besitzt eine Festplatte eine
Steuerelektronik, mit der sie gesteuert wird. Im Detail übernimmt sie die Steuerung des Platter-Motors sowie die Steuerung und Positionierung des
Schreib-/Lesearms, die Verarbeitung der gelesenen und zu schreibenden Daten und die Protokollumsetzung zum Anschluss an ein Rechnerbussystem. Dieser Sachverhalt
ist in Abbildung \ref{img:hdd1} skizzenhaft dargestellt.

Die Platter sind in mehrere konzentrische Spuren unterteilt, die als Zylinder bezeichnet werden. Bei den Zylindern handelt es sich nicht, wie möglicherweise
vermutet werden kann, um Vertiefungen im Platter sondern der Lese-/Schreibkopf schwebt im Abstand von wenigen Nanometern über ihnen. Das Schweben wird hierbei
durch den \textsc{Bernoulli}-Effekt ereicht. Die Zylinder sind ihrerseits weiter in Blöcke unterteilt, wobei diese nicht nur Nutzdaten enthalten. Vor den
Nutzdaten -- meist 512kB -- befindet sich eine Präambel, welche zur Kalibrierung des Lese-/Schreibkopfes dient. Dahinter befindet sich ein \ac{ECC}, welcher zur
Fehlererkennung und Korrektur beim Lesen der Daten dient (Abbildung \ref{img:hdd2}). Die Blöcke, die unter dem gleichen radialen Winkel vom Schreib-/Lesekopf
erreicht werden können, werden auch Sektor genannt. Dadurch lässt sich ein beliebiger Block durch Angabe des Sektors, Zylinders und Kopfs adressieren. Die Anzahl
von Sektoren, Zylindern, Köpfen und Größe der Blöcke wird hierbei auch als Festplattengeometrie bezeichnet, mit deren Hilfe sich die Kapazität der Festplatte wie
folgt berechnen lässt:

\begin{center}
$\text{Kapazität} = \text{Köpfe} \cdot \text{Sektoren} \cdot \text{Zylinder} \cdot \text{Blockgröße}$
\end{center}

\begin{figure}[b!]\centering
    \includetikz[0.95]{figures/chapter2/hdd2}%
    \caption[Schematische Darstellung eines Platters]{Schematische Darstellung eines Platters (nach \textcite[100]{tanenbaum1})}
    \label{img:hdd2}
\end{figure}

Bei modernen Festplatten wird die Geometrie der Festplatte durch den Festplattencontroller jedoch vor dem übrigen System verborgen. Ursprünglich hatte dies als
Ursache, dass die Adressierung bestimmten Limitierungen bezüglich der maximalen Anzahl der Köpfe, Sektoren und Zylinder unterworfen war. Die Kapazität der
Festplatten wurde dadurch erhöht, dass pro Platter mehr Spuren und mehr Sektoren pro Spur untergebracht wurden. Die Anzahl der Platter veränderte sich jedoch
nicht signifikant. Dies führte dazu, dass bald die Anzahl der Sektoren und Zylinder die der maximal adressierbaren Anzahl von Sektoren und Zylinder überstieg. Um
dieses Problem zu umgehen, wurde dem restlichen System durch den Festplattencontroler eine größere Anzahl von Köpfen gemeldet, als real vorhanden waren. Dies
erweiterte die Anzahl der adressierbaren Sektoren deutlich.

Im weiteren Verlauf wurde die Aufteilung des Platters weiter verändert, so dass bei modernen Festplatten im äußeren Bereich eines Platters mehr Sektoren
untergebracht sind als im inneren Bereich. Weiter geht die Tendenz aktuell dahin, dass die Nettoblockgröße von 512 Byte auf vier Kilobyte erhöht 
wird~\cite[vgl.][]{wdc}, um zusätzlichen Platz einzusparen, der durch Präambel, \ac{ECC} und Lücke zwischen den Blöcken belegt wird. Aus diesen Gründen wurde die
Adressierung durch Kopf, Sektor und Zylinder durch eine fortlaufende Nummer ersetzt. Diese wird vom Festplattencontroller in die festplattenspezifischen
Geometrieparameter übersetzt.

Aus den genannten physikalischen Eigenschaften ergeben sich Vorteile aber auch Nachteile. Der größte Vorteil einer Festplatte gegenüber fast allen anderen
Speichermedien ist die Möglichkeit, große Datenmengen mit einem wahlfreien Zugriff zu einem sehr niedrigen Preis zu speichern. Weiterhin können die Daten relativ
schnell sequenziell gelesen und geschrieben werden, so dass Festplatten ideal sind für eine mittelfristige Datenarchivierung. Festplatten werden jedoch nicht
ausschließlich hierfür eingesetzt sondern auch als Systemspeicher, in dem Betriebssystem, Programme und sonstige Nutzerdaten gespeichert werden. Genau hier
werden die Nachteile von Festplatten sichtbar. Diese Daten werden nicht sequenziell sondern hauptsächlich ungeordnet gelesen. Dies ist problematisch, da die
Leserate von Festplatten beim ungeordneten Lesen dramatisch einbricht. Der Grund hierfür ist im Aufbau zu finden. Um zu einem bestimmten Block zu springen, muss
zuerst der Lese-/Schreibarm in die richtige Position gebracht werden. Diese Verzögerung wird \textit{Seek Time} genannt und liegt bei aktuellen Festplatten
zwischen ca. drei und zwölf Millisekunden~\cite{seagate1,samsung1}. Außerdem braucht der Platter im Durchschnitt eine halbe Umdrehung, bis der
angeforderte Block unter dem Lese-/Schreibkopf liegt. Diese Verzögerung wird \textit{Latency} genannt. Hinzu kommt noch die Verzögerung des
Festplattencontrollers (\textit{Controller-Overhead}), die aber im Vergleich zu den anderen beiden Größen recht gering ist. Daher lässt sich die mittlere
Zugriffszeit für einen beliebigen Block wie folgt berechnen:

\begin{center}
$\text{Zugriffszeit[ms]} = \text{Spurwechselzeit[ms]} + \underbrace{\frac{60000}{\text{Drehzahl[min}^{-1}\text{]}} \cdot 0,5}_{\text{Latency}} +
\text{Controller-Overhead[ms]}$
\end{center}

Um das eben beschriebene Problem beim unsortierten Lesen zu mindern, besitzen moderne Festplatten einen mehrere Megabyte großen Halbleiterspeicher-basierten
Lese-/Schreib\-cache. Dieser ermöglicht es den Festplatten, häufig angefragte Blöcke vorzuhalten und Lese-/Schreiboperationen ggf. umzuordnen. Das Umordnen kann
z.B. sinnvoll sein, wenn ein Block im Anfragepuffer liegt, der sich gerade unter dem Lese-/Schreibkopf befindet. Es ist sinnvoller, zunächst diese
Anfrage zu bearbeiten als später wieder den Lese-/Schreibkopf zu dieser Position zu bewegen.

\section{Solid State Drive (SSD)}
\label{chap2:ssd}

Der Aufbau und die Funktionsweise von \acp{SSD} wird von \textcite{anand1,anand2} diskutiert. Ein \ac{SSD} dient zwar genau wie eine herkömmliche
magnetscheibenbasierte Festplatte als Massenspeicher, ist jedoch fundamental anders aufgebaut. Bei einem \ac{SSD} werden die Daten nicht in Form einer
magnetischen Ausrichtung gespeichert sondern als elektrische Ladung. Dies wird durch eine Implementierung in Halbleitertechnik ermöglicht.

Die Grundbausteine eines \ac{SSD} sind Flash-Speicherbausteine. Der Flashspeicher ist ein persistenter Halbleiterspeicher. Es wird eine im Vergleich zum
restlichen System relativ hohe Spannung zum Programmieren bzw. Löschen benutzt. Diese Spannung beim Programmieren des Flashspeicher-Transistors macht ihn
dauerhaft nicht-leitend. So ist es möglich, beim normalen Schalten des Transistors zwischen programmiert und nicht programmiert bzw. gelöscht zu unterscheiden.
Außerdem wird durch diese Technik ein schnelles Auslesen ermöglicht. Das Programmieren mit hoher Spannung hat jedoch den Nachteil, dass es den Transistor
physisch schädigt und somit nur eine begrenzte Menge von Programmiervorgängen möglich sind. Weiterhin werden Flashspeicherzellen in \ac{SLC}- und
\ac{MLC}-Speicherzellen unterschieden. Bei \ac{SLC}-Speicherzellen wird nur ein Bit pro Flashzelle gespeichert, bei \ac{MLC}-Speicherzellen zwei oder mehr.
\ac{MLC}-Speicherzellen haben einen Kostenvorteil, da sie kleiner sind und somit günstiger bei der Herstellung. Sie haben jedoch den Nachteil, langsamer beim
Lesen und Programmieren und Fehleranfälliger als \ac{SLC}-Speicherzellen zu sein.

\begin{figure}[b!]\centering
  \vspace{0.5cm}
	\includetikz[1.25]{figures/chapter2/ssd1}%
	\caption[Zeitlicher Ablauf des Schreibens auf einen Cacheblock mit ungültigen Seiten]{Zeitlicher Ablauf des Schreibens auf einen Cacheblock mit ungültigen Seiten (nach \textcite{anand1})}
	\label{img:ssd1}
\end{figure}

Auf Grund des zu hohen Aufwandes bei der Adressierung einzelner Flash-Speicherzellen werden diese heute für gewöhnlich zu Seiten à vier Kilobyte zusammengefasst.
Somit ist die kleinstmögliche schreibbare Datenmenge bei einem Flashbaustein vier Kilobyte groß. Da der Aufwand beim Aufbau zum Löschen der Speicherzellen auf
Grund der benötigten hohen Spannung noch größer ist, werden für gewöhnlich wiederum 128 der vier Kilobyte Seiten zu einem Löschblock zusammengefasst. Somit ist
die kleinste löschbare Einheit 512kB groß. Das bedeutet, dass zum Ändern des Inhalts einer Seite zunächst die anderen 127 Seiten ausgelesen und anschließend
zusammen mit der geänderten Seite wieder zurückgeschrieben werden müssen. Das Vorgehen wird im zeitlichen Ablauf in Abbildung \ref{img:ssd1} dargestellt, wobei
ein Löschblock der Übersichtlichkeit halber nur aus neun Seiten besteht.

\begin{figure}[b!]\centering
	\includetikz[1.05]{figures/chapter2/ssd2}%
	\caption[Aufbau eines SSD]{Aufbau eines SSD (nach \textcite{anand1})}
	\label{img:ssd2}
\end{figure}

Um die Kompatibilität zu den bisherigen Festplatten zu wahren und den Flash-Speicher an einen \mbox{S-ATA}-Controler, SAS-Controller oder den PCI-Express-Bus
anzuschließen, wird ein weiterer Controllerchip benötigt. Neben der Aufgabe der Protokollwandlung übernimmt dieser das so genannte \textit{Wear-Levelling}.
Wear-Levelling soll die Lebenszeit von Flashspeichermedien verlängern. Dies wird dadurch erreicht, dass Schreibzugriffe möglichst gleichmäßig auf alle Seiten
verteilt werden. Ein typisches Szenario hierfür sind die Verwaltungsinformationen des Dateisystems. Sie liegen immer an derselben logischen Speicheradresse und
werden regelmäßig verändert. Ohne Wear-Levelling würde immer wieder dieselbe Seite beschrieben werden und relativ schnell ausfallen, während die meisten anderen
noch ihre nahezu vollständige Lebenszeit hätten. Wear-Levelling sorgt nun dafür, dass das Schreiben nicht immer wieder auf derselben Seite stattfindet sondern
auf einer, auf der in der Vergangenheit weniger Schreibvorgänge stattgefunden haben. So werden sämtliche Schreibvorgänge gleichmäßig auf alle Seiten des
\ac{SSD}s verteilt. Dies bedeutet allerdings auch einen nicht zu unterschätzenden Verwaltungsaufwand. Zum einen muss die Zuordnung der logischen Plattenadresse
auf eine konkrete Flashseite vorgenommen und gespeichert werden und zum anderen die Anzahl der Schreibzyklen pro Seite gespeichert werden. Der grundsätzliche
Aufbau eines \ac{SSD}s wird zum besseren Verständnis in Abbildung \ref{img:ssd1} schematisch dargestellt, wobei die Darstellung nur auf Baugruppen eingeht, die
für die Programmierung von Relevanz sind.

Ein Problem, das bei der Substitution von Festplatten durch \acp{SSD} entsteht, liegt in der unterschiedlichen Funktionsweise beim Löschen von Daten. Daten auf
Festplatten können ohne vorheriges Löschen blockweise überschrieben werden. Bei \acp{SSD} ist dies nicht der Fall. Hier müssen Blöcke vor dem Neubeschreiben
zuerst gelöscht werden. Dies wirft zunächst das bereits oben beschriebenen Problem auf, dass sich sowohl gültige als auch ungültige Daten in einem Block befineen
können. Wenn die ungültigen Blöcke überschrieben werden sollen, muss erst, wie oben gezeigt, der komplette Block ausgelesen, verändert und dann zurückgeschrieben
werden. Da das Betriebssystem kein Wissen über die Löschblockgröße eines \ac{SSD} hat, ist es der Regelfall, dass Daten mehrerer Dateien in einem Block
untergebracht werden, wordurch das gennante Problem sehr häufig auftritt.

\begin{figure}[b!]\centering
	\includetikz[0.8]{figures/chapter2/ssd3}%
	\caption[Schema zur Bereitstellung leerer SSD-Blöcke]{Schema zur Bereitstellung leerer SSD-Blöcke (nach \textcite{anand2})}
	\label{img:ssd3}
\end{figure}

Dies ist solange relativ unkritisch, solange leere Blöcke zum Bedienen von Schreibanfragen zur Verfügung stehen. Sobald jedoch zwangsweise Daten in die
ungültigen Bereiche von teilweise gelöschten Blöcken geschrieben werden müssen, kann ein Leistungseinbruch beobachtet werden. Das Vorgehen eines \ac{SSD} beim
Schreiben ist in Abbilung \ref{img:ssd3} schematisch dargestellt. Solange sich genügend Blöcke in der Liste der leeren Blöcke befinden, kann das \ac{SSD} mit
voller Geschwindigkeit arbeiten. Sobald der Vorrat jedoch erschöpft ist, müssen aus dem oben genannten Grund teilweise gefüllte Blöcke geräumt werden, der Block
gelöscht und die geräumten Daten erneut in einen leeren Block geschrieben werden. Dieses Vorgehen ist zeitaufwendig und führt zu dem beobachteten Phänomen des
Leistungsverlustes bei der längeren Nutzung eines \ac{SSD}.

Das Problem könnte umgangen werden, indem die Daten z.B. während der Leerlaufzeiten des \ac{SSD}, die im normalen Desktopbetrieb relativ häufig auftreten,
umsortiert werden könnten. Somit würden fast immer ausreichend leere Blöcke zur Verfügug stehen, solange das \ac{SSD} nicht komplett gefüllt ist. Dies ist jedoch
nicht ohne weiteres möglich, da die Betriebssysteme bzw. genauer die genutzten Dateisysteme unter der Prämisse entwickelt wurden, dass ungültige Daten ohne
Leistungsverlust überschrieben werden können, wie es bei herkömmlichen Festplatten auch der Fall ist. Dateisysteme markieren beim Löschen von Dateien lediglich
die zuvor belegten Blöcke als ungültig in ihrem Inhaltsverzeichnis und geben diese Information nicht an das Blockgerät weiter. Bei \acp{SSD} ist dies von
erheblichem Nachteil, da das Gerät erst beim Überschreiben eines Blockes von seiner Ungültikeit erfährt und das oben beschriebene Umsortieren erst stattfinden
kann, wenn der Block bereits wieder gebraucht wird und der Vorgang somit zeitkritisch ist. Um diesem Problem entgegenzutreten wurde vor kurzem der
\mbox{S-ATA}-Standard um den so genannten \texttt{TRIM}-Befehl erweitert~\cite{acs2}. Dieser ermöglicht es, \mbox{S-ATA}-basierte Geräte über die
Ungültigkeit von Blöcken zu informieren. Unglücklicherweise müssen jedoch für die Nutzung sowohl das Dateissystem als auch Betriebssystemkomponenten wie
Gerätetreiber und Treiberschnittstellen angepasst werden. So kommt es, dass es momentan zwar schon Geräte gibt, die diese Möglichkeit unterstützen, sie jedoch
auf Grund der mangelnden Softwareunterstützung nicht genutzt werden kann.