\chapter{Einführung}
\label{chap1}

Heutige Rechensysteme haben einen Stand erreicht, in dem pro Sekunde mehrere Gigabyte Daten verarbeitet werden können. Diese Entwicklung bringt jedoch das
Problem mit sich, dass die Daten dem Prozessor schnell genug zur Verfügung gestellt werden müssen. Zu Beginn der elektronischen Datenverarbeitung hatten
Prozessoren eine Verarbeitungsgeschwindigkeit, die mit den Leseraten der damaligen Speichermedien wie z.B. Lochkarten gleichzusetzen waren. Auf Grund zunehmender
Miniaturisierung entwickelten sich jedoch die Prozessoren rasant weiter. Sie folgten hierbei dem \textsc{Moore}schen Gesetz, welches besagt, dass sich die
Komplexität integrierter Schaltkreise alle 18 Monate verdoppelt und dadurch auch die Rechenleistung pro Chipfläche exponentiell zunimmt.

Im Gegensatz zur schnellen Entwicklung auf dem Prozessormarkt stand  die Entwicklung im Bereich der Speichertechnik. Die Lochkarten wurden durch schneller
lesbare Magnetbänder verdrängt und diese wiederum später durch Magnetplatten bzw. Festplatten ersetzt. Dies führte zum einen zu einer enormen Platzersparnis, zum
anderen zu einem Geschwindigkeitsgewinn beim Auslesen der Daten. Zwar konnten die Prozessoren mit ausreichend Daten im Sinne der Menge versorgt werden,
aber die Entwicklung im Bereich der Speichertechnik konnte nicht mit dem exponentiellen Wachstum der Prozessorgeschwindigkeit mithalten. Dies führte zu großen
Problemen, weil die zu verarbeitenden Daten nicht schnell genug für die Verarbeitung durch den Prozessor zur Verfügung gestellt werden konnten.

Neben der Speichermöglichkeit auf Lochkarten und magnetischen Medien existierte weiterhin die Möglichkeit, Speicher in Form von Halbleitern zu realisieren. In
der Praxis hatte dies jedoch die Nachteile, dass selbst kleine Speichermengen extrem teuer waren und Daten sich nur flüchtig speichern ließen. Deshalb kamen
Halbleiterspeicher als Massenspeicher nicht in Frage. Allerdings lässt sich bereits mit einer kleinen Menge dieser Art von Speichern die Eigenschaft der
Lokalität der meisten Programme ausnutzen. Lokalität bedeutet hierbei, dass Programmteile häufig wieder verwendet werden und sich ggf. nur die zu bearbeitenden Daten
ändern. Genau diese Daten lassen sich gut im flüchtigen Halbleiterspeicher halten, der heute allgemein als Haupt- oder Arbeitsspeicher bezeichnet wird. Dies
funktionierte zunächst sehr gut, da der Speicher schneller war als die verfügbaren Prozessoren, jedoch entwickelten sich nachfolgend auch hier die Prozessoren
schneller weiter als der Speicher, wie Abbildung \ref{chap1:ram-cpu} zeigt. Im Diagramm wird die Geschwindigkeit von Prozessoren mit dem hauptsächlich für
Arbeisspeicher verwendeten Halbleiterspeicher DRAM verglichen.

\newpage

\begin{figure}[t!]\centering
    \includetikz[0.75]{figures/chapter1/ram-cpu}%
    \caption[Entwicklung der Geschwindigkeit von Prozessoren und Speicher]{Entwicklung der Geschwindigkeit von Prozessoren und Speicher (nach~\textcite[180]{intro1})}
    \label{chap1:ram-cpu}
\end{figure}

\begin{figure}[b!]\centering
    \includetikz[0.85]{figures/chapter1/hierarchy_default}%
    \caption[Speicherhierarchie]{Speicherhierarchie (nach~\textcite[45]{stallings:os})}
    \label{chap1:mem1}
\end{figure}

Das Problem der auseinanderdriftenden Geschwindigkeiten, das im Diagramm deutlich sichtbar wird, ist jedoch nicht primär technologischer sondern ökonomischer
Natur. Halbleiterspeicher profitieren ebenso wie Prozessoren von der steigenden Integrationsdichte. Dabei muss beachtet werden, dass die Kosten für die
Produktion proportional mit der Geschwindigkeit des Speichers steigen. Es ist zwar denkbar, Arbeitsspeicher mit ausreichender Geschwindigkeit zu produzieren,
dies war und ist ökonomisch jedoch nicht sinnvoll. Vielmehr führte es dazu, dass zusätzlich zum Arbeitsspeicher ein weiterer Halbleiterspeicher mit einer
geringeren Kapazität aber besserer Leistung und anderer Funktionalität in Rechnersysteme integriert wurde. Dieser als Cache bezeichnete Speicher wird, anders
als der Arbeitsspeicher, nicht vom Benutzer verwaltet sondern von der Hardware selbst. Hierbei sollte angemerkt werden, dass es sich meistens bei dem Benutzer nicht
direkt um den Endnutzer sondern meist um das Betriebssystem handelt. Folglich ist der Cache dem Benutzer gegenüber vollkommen transparent. Da das
Konzept eines einfachen Caches zu einem bestimmten Zeitpunkt auch nicht mehr ausreichte, wurden mehrere Cachestufen mit unterschiedlichen Geschwindigkeiten
integriert, so dass es zu einer Speicherhierarchie kam, wie sie in Abbildung \ref{chap1:mem1} zu sehen ist. Dabei kann festgestellt werden, dass der Cache immer
wesentlich kleiner ist als der gecachte Speicher, dafür aber für gewöhnlich um Größenordnungen schneller.

In Abbildung \ref{chap1:mem1} werden neben den bereits aufgeführten Speicherarten optische Speichermedien genannt. Diese setzten sich Anfang der 1990er Jahren
zum Transport von Daten zwischen Rechensystemen und zum Anfertigen von Sicherheitskopien neben Bandlaufwerken durch. Des Weiteren gilt es zu beachten, dass in
der Abbildung ein Trade-off zwischen Speicherplatz und Kosten auf der einen Seite und Datendurchsatz und Zugriffszeit auf der anderen Seite zu erkennen ist,
welcher vertikal in der Hierarchie verläuft. Das Prinzip dieser Speicherhierarchie hat sich seit längerem durchgesetzt und stellt die momentan übliche
Herangehensweise bei der Architektur eines Rechners dar. Die Hierarchie weist jedoch ein Defizit auf, welches bis in die heutige Zeit zunehmend markanter wurde.
Es besteht eine Lücke bei der Zugriffszeit zwischen Halbleiterspeicher und Magnetspeicher. Diese ist mehrere Dekaden groß, wie in Abbildung \ref{img:gab} zu
erkennen ist.

\begin{figure}[t!]\centering\vspace{5mm}
    \includetikz{figures/chapter1/memory_technologies}%
    \caption[Vergleich gebräuchlicher Speichertechnologien]{\mbox{Vergleich gebräuchlicher Speichertechnologien (nach \textcite[245]{info2})}}
    \label{img:gab}
\end{figure}

Neben dem bereits erwähnten flüchtigen Halbleiterspeicher wurden in den 1980er Jahren nichtflüchtige Halbleiterspeicher entwickelt. Da diese jedoch wesentlich
langsamer als flüchtige Halbleiterspeicher sind, können sie nicht als Ersatz für die flüchtigen Hauptspeicher dienen. Des Weiteren hatte er ökonomisch ähnliche
Nachteile wie der flüchtige Speicher, so dass er ebenfalls nicht als Ersatz für Magnetscheiben basierte Festplatten dienen konnte. Er fand zunächst seine
Anwendung in eingebetteten Geräten, setzte sich jedoch Mitte der 1990er Jahre durch die aufkommende Digitalfotografie auch als Wechselspeichermedium durch, da
hierbei auf Grund der Größe und physikalischen Belastungen weder optische noch magnetische Speichermedien geeignet waren. Der hierfür verwendete
Halbleiterspeicher wird als Flashspeicher bezeichnet. Neben den hauptsächlich für die Digitalfotografie entwickelten Speicherkarten kamen Ende der 1990er Jahre
USB-basierte Flashspeichermedien auf den Markt, die bis heute im Begriff sind, optische Speichermedien für den Datentransport zu ersetzen.

In den letzten Jahren gab es allerdings eine neue Entwicklung. Da Flashspeicher auch dem \textsc{Moore}schen Gesetz unterliegen, sind die Preise pro Byte
schnell gesunken und die Leistung gleichzeitig gestiegen. Aus diesem Grund sind seit ein paar Jahren Flashspeicher basierte Festplatten - so genannte \acp{SSD} - auf dem
Markt. Also lässt sich die Speicherhierarchie zunächst wie in Abbildung \ref{chap1:mem2} gezeigt erweitern.

\begin{figure}[t]\centering
	\subfigure[]{
		\includetikz[0.85]{figures/chapter1/hierarchy_new1}%
		\label{chap1:mem2}
	}
	\hfill
	\subfigure[]{
		\includetikz[0.85]{figures/chapter1/hierarchy_new2}%
		\label{chap1:mem3}
	}
	\caption[Erweiterte Speicherhierachien]{Erweiterte Speicherhierachien}
\end{figure}

\acp{SSD} haben jedoch andere Eigenschaften als klassische Festplatten. Sie sind zwar nur moderat schneller in Bezug auf den Datendurchsatz jedoch um
Größenordnungen schneller in Bezug auf die Zugriffszeit. Auf Grund dieser Eigenschaft können sie die technologische Lücke, wie sie in Abbildung \ref{img:gab}
dargestellt ist, schließen. Wenn nun noch vor Augen geführt wird, dass auf bestimmte Arten von Daten wie z.B. Programmdaten hauptsächlich ungeordnet
zugegriffen wird, kann leicht erkannt werden, dass \acp{SSD} Festplatten in diesem Bereich signifikant überlegen sind. Unglücklicherweise übersteigen die Kosten
pro Byte \ac{SSD}-Speicher die Kosten pro Byte Festplattenspeicher noch immer signifikant. Dies ist deutlich ablesbar in Tabelle \ref{chap1:preise}, in der
die Preise gängiger Low- bis High-End-Festplatten mit Preisen von Low- bis High-End-\acp{SSD} verglichen werden. Aus diesem Grund kann auf Festplatten noch
nicht komplett verzichtet werden.

\begin{table}[b]\centering
  \caption{Preise ausgewählter Festplatten und SSDs (Quelle: Geizhals Preisvergleich, Stand: 13.05.2010)} 
	\begin{tabular}{l|r}
	Produkt & Preis pro TB in Euro \\ \hline
	\multicolumn{2}{c}{\textbf{Festplatten}} \\ \hline
	Samsung EcoGreen F2 1500GB (5400rpm) & 48,30 \\ \hline
	Seagate Barracuda 7200.11 1500GB (7200rpm) & 54,32 \\ \hline
	Western Digital VelociRaptor 600GB (10000rpm) & 431,65 \\ \hline
	Hitachi Ultrastar 15K600 600GB (15000rpm) & 665,00 \\ \hline
	\multicolumn{2}{c}{\textbf{\acp{SSD}}} \\ \hline
	BestMedia Platinum MyDrive 128GB & 1373,18 \\ \hline
	Kingston SSDNow V-Series G2 128GB & 1796,10 \\ \hline
	Intel X25-M G2 Postville 160GB & 2295,80 \\ \hline
	Intel X25-E 64GB & 9967,62 \\  \hline
	\end{tabular}
	\label{chap1:preise}
\end{table}

\newpage

Um dennoch die Möglichkeiten und Vorteile der neuen Technologie nutzen zu können, muss der Nutzer die Daten, auf die hauptsächlich sequenziell
zugegriffen wird, von den Daten manuell trennen, auf die hauptsächlich ungeordnet zugegriffen wird, und auf die entsprechend passenden Speichermedien verteilen.
Da dies sehr umständlich und für unerfahrene Nutzer kaum machbar ist, wäre es sinnvoller \acp{SSD} nicht als kompletten Ersatz für herkömmliche Festplatten zu
nutzen, sondern als Cache, genau wie es bei Hauptspeicher und (Prozessor)-Cache der Fall ist. Es würde sich dadurch eine veränderte Speicherhierarchie
ergeben, wie sie in Abbildung \ref{chap1:mem3} zu sehen ist.

Diese Arbeit soll sich mit der Möglichkeit der Realisierung dieses Caches auseinandersetzen. Hierfür wird zunächst in Kapitel \ref{chap2} das technische Umfeld
näher untersucht und dabei die Funktionsweise von Festplatten und \acp{SSD} erläutert. Das darauf folgende Kapitel \ref {chap3} setzt sich mit Arbeiten
auseinander, die für diese Arbeit relevant sind. Anschließend wird in Kapitel \ref{chap4} eine Problemstellung mit konkreten Anforderungen formuliert. Sodann
wird in Kapitel \ref{chap5} die Architektur des implementierten Caches vorgestellt und die Implementierung in Kapitel \ref{chap6} näher erläutert. Anschließend
wird in Kapitel \ref{chap7} die Leistungsfähigkeit des Caches untersucht und ein Ausblick auf zukünftige Arbeitsfelder in Kapitel \ref{chap8} gegeben.
